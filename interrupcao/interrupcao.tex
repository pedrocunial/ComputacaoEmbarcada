% Created 2017-03-29 Wed 11:37
\documentclass[11pt]{article}
\usepackage[utf8]{inputenc}
\usepackage[T1]{fontenc}
\usepackage{fixltx2e}
\usepackage{graphicx}
\usepackage{grffile}
\usepackage{longtable}
\usepackage{wrapfig}
\usepackage{rotating}
\usepackage[normalem]{ulem}
\usepackage{amsmath}
\usepackage{textcomp}
\usepackage{amssymb}
\usepackage{capt-of}
\usepackage{hyperref}
\author{Pedro Cunial Campos}
\date{\today}
\title{10 Pesquisa -- Interrupção e Exceções}
\hypersetup{
 pdfauthor={Pedro Cunial Campos},
 pdftitle={10 Pesquisa -- Interrupção e Exceções},
 pdfkeywords={},
 pdfsubject={},
 pdfcreator={Emacs 25.1.1 (Org mode 8.3.6)},
 pdflang={English}}
\begin{document}

\maketitle
\tableofcontents



\section{Introdução}
\label{sec:orgheadline1}
\section{Exceções}
\label{sec:orgheadline3}
\subsection{Qual a diferença entre as exceções NMI e IRQ?}
\label{sec:orgheadline2}
As exceções NMI (Non-maskable Interrupt) que não podem ser ignoradas
por técnicas comuns de mascaramento de interrupções, ou seja, são interrupções
de maior prioridade, comumente utilizadas em caso de erros críticos que podem
gerar estados não recuperáveis do software ou hardware caso não sejam tratadas
com urgência.

Enquanto isso, as IRQ (Interrupt Request) são interrupções de menor prioridade
que, como o nome já sugere, adiciona um pedido de interrupções na stack de um
manipulador de interrupções do sistema. IRQ costumam ser utilizadas para a
manipulação de eventos, sejam eles de recebimentos de dados, ou de teclas
precionadas.

\section{Interrupção}
\label{sec:orgheadline11}
\subsection{Qual a diferença entre exceções IRQ e ISR?}
\label{sec:orgheadline4}
Enquanto as IRQ inserem requesições de interrupções direto na stack do
manipulador de eventos, enquanto as ISR são tratadas pelo seu respectivo
interruption handler.
\subsection{No ARM que utilizamos no curso, quantas são as interrupções suportadas e qual a sua menor prioridade?}
\label{sec:orgheadline5}
O NVIC do ARM utilizado no curso suporta 72 interrupções diferentes, sendo o
menor nível de prioridade possível 7 (8 níveis diferentes de prioridade).
\subsection{Descreva o uso do FIQ.}
\label{sec:orgheadline6}
A FIQ é uma interrupção de altíssima prioridade que desabilita qualquer outro
handler (seja o handler comum de eventos e interrupções ou o handler específico
de uma ISR ou outra FIQ).
\subsection{No diagrama anterior, quem possui maior prioridade IRQ ou FIQ?}
\label{sec:orgheadline7}
Pelo diagrama, o FIQ tem maior prioridade em comparação ao IRQ, o que pode ser
explicado pelo fato das FIQs desabilitarem qualquer outro handler ao enquanto
acionadas.
\subsection{No datasheet, secção 13.1 informa o ID do periférico que está associado com a sua interrupção. Busque a informação e liste o ID dos seguintes periféricos:}
\label{sec:orgheadline8}
\begin{itemize}
\item PIOA

10

\item PIOC

12

\item TC0

23
\end{itemize}

\subsection{O que aconteceria caso não limpemos a interrupção?}
\label{sec:orgheadline9}
Caso não limpemos a interrupção, o NVIC não saberá que a interrupção fora
resolvida e, portanto, ficará "preso" no seu processo, podendo apenas receber
interrupções de maior prioridade, e sem voltar para a thread principal da sua
aplicação.
\subsection{O que é é latência na resolução de uma interrupção, o que é feito nesse tempo? (Interrupt Latency)}
\label{sec:orgheadline10}
Em suma, a latência na resolução de uma interrupção é o tempo entre o envio da
interrupção e a execução do código associado a mesma. Dependendo do tipo de
interrupção, existem diferentes processos que podem ocorrer entre a mesma e
sua execução, por exemplo, no caso de interrupções de menor prioridade, o
intervalo entre seu sinal e o início de sua execução pode ser utilizado pela
execução de uma interrupção de maior prioridade. Em casos de IRQ, por exemplo,
sua latência costuma também estar associada à eventos "à sua frente" na stack
do manipulador de eventos.

\section{Software - CMSIS}
\label{sec:orgheadline12}

\section{PIO - Interrupção}
\label{sec:orgheadline16}
\subsection{Qual deve ser a configuração para operarmos no botão do kit SAME70-EK2?}
\label{sec:orgheadline13}
Para operarmos um botão pela sua interrupção, devemos utilizar um handler
associado ao mesmo, como feito no exemplo da aula 10. O trecho de código abaixo
exemplifica seu uso. Repare na função pio\(_{\text{handler}}_{\text{set}}\), é nela que associamos ao
botão o seu handler.

\begin{verbatim}
/**
* @Brief Inicializa o pino do BUT
*  config. botao em modo entrada enquanto
*  ativa e configura sua interrupcao.
*/
void but_init(
    Pio *p_but_pio,
    const u_int32_t pio_id,
    const u_int32_t but_pin_mask)
{
  /* config. pino botao em modo de entrada */
  pmc_enable_periph_clk(pio_id);
  pio_set_input(p_but_pio, but_pin_mask, PIO_PULLUP | PIO_DEBOUNCE);

  /* config. interrupcao em borda de descida no botao do kit */
  /* indica funcao (but_Handler) a ser chamada quando houver uma interrupção */
  pio_enable_interrupt(p_but_pio, but_pin_mask);
  switch (but_pin_mask) {
    case BUT_PIN_MASK:
      pio_handler_set(p_but_pio, pio_id, but_pin_mask, PIO_IT_FALL_EDGE, but_Handler);
      break;
    case BUT1_PIN_MASK:
      pio_handler_set(p_but_pio, pio_id, but_pin_mask, PIO_IT_RISE_EDGE, but1_Handler);
      break;
    case BUT2_PIN_MASK:
      pio_handler_set(p_but_pio, pio_id, but_pin_mask, PIO_IT_FALL_EDGE, but2_Handler);
      break;
    case BUT3_PIN_MASK:
      pio_handler_set(p_but_pio, pio_id, but_pin_mask, PIO_IT_RE_OR_HL, but3_Handler);
      break;
  }

  /* habilita interrupçcão do PIO que controla o botao */
  /* e configura sua prioridade                        */
  NVIC_EnableIRQ(pio_id);
  NVIC_SetPriority(pio_id, 1);
}

void but_Handler()
{
  /*
  *  limpa interrupcao do PIO
  */
  uint32_t pioIntStatus;
  pioIntStatus =  pio_get_interrupt_status(BUT_PIO);
  /**
    *  Toggle status led
    */
  if(pio_get_output_data_status(LED_PIO, LED_PIN_MASK))
    pio_clear(LED_PIO, LED_PIN_MASK);
  else
    pio_set(LED_PIO, LED_PIN_MASK);
}
\end{verbatim}

\subsection{Com base no texto anterior e nos diagramas de blocos descreva o uso da interrupção e suas opções.}
\label{sec:orgheadline14}
Um PIO Controller pode gerar uma interrupção em dadas variações de seu
respectivo valor, por exemplo, podemos associar uma interrupção ao pressionar
um dado botão, fazendo com que o uC só seja de fato requisitado no evento do
pressionar deste dado botão.

Da mesma forma, podemos associar eventos ao pressionar e ao soltar de um dado
botão, podendo reproduzir o código de acender o LED enquanto o botão estiver
pressionado utilizando apenas eventos de pressionar e soltar o botão, mantendo
o uC em sleep mode (reduzido gasto energético).
\subsection{Descreva as funções dos registradores:}
\label{sec:orgheadline15}
\begin{itemize}
\item PIO\(_{\text{IER}}\) / PIO\(_{\text{IDR}}\)

Como o próprio nome já sugere, o PIO Interrupt Enable/Disable Register,
respectivamente, habilitam ou desabilitam o uso de interrupções associadas ao
dado PIO.

\item PIO\(_{\text{AIMER}}\) / PIO\(_{\text{AIMDR}}\)

O PIO Additional Interrupt Modes Enable/Disable Register são utilizados para
definir opções adicionais possíveis para as interrupções associadas ao dado
PIO.

\item PIO\(_{\text{ELSR}}\)

O PIO Edge/Level Select Register é utilizado para definir se a interrupção
será enviada em uma borda (subida ou descida), ou de acordo com o seu nível
(alto ou baixo).

\item PIO\(_{\text{FRLHSR}}\)

O PIO Fall/Rise Low/High Status Register serve para definir se o PIO deve
agir na borda de subida ou descida (no caso de estar definido como
trabalhando nas bordas) ou no sinal alto ou baixo (no caso de ser definido
pelo nível).
\end{itemize}
\end{document}