% Created 2017-02-22 Wed 16:00
\documentclass[11pt]{article}
\usepackage[utf8]{inputenc}
\usepackage[T1]{fontenc}
\usepackage{fixltx2e}
\usepackage{graphicx}
\usepackage{grffile}
\usepackage{longtable}
\usepackage{wrapfig}
\usepackage{rotating}
\usepackage[normalem]{ulem}
\usepackage{amsmath}
\usepackage{textcomp}
\usepackage{amssymb}
\usepackage{capt-of}
\usepackage{hyperref}
\author{Pedro Cunial}
\date{\today}
\title{Pesquisa 07 -- PIO Output}
\hypersetup{
 pdfauthor={Pedro Cunial},
 pdftitle={Pesquisa 07 -- PIO Output},
 pdfkeywords={},
 pdfsubject={},
 pdfcreator={Emacs 25.1.1 (Org mode 8.3.6)},
 pdflang={English}}
\begin{document}

\maketitle
\tableofcontents


\section{Periféricos}
\label{sec:orgheadline4}
\subsection{Liste a funcionalidade dos periféricos listados a seguir:}
\label{sec:orgheadline1}
\begin{itemize}
\item RTC -- Real Time Clock

O RTC, assim como seu nome sugere, é um periférico capaz de manter informações sobre tempo atual (segundo,
hora, dia etc) e forncê-las. O RTC possui uma memória SRAM, normalmente de 56 bytes e é capaz de notificar um
possível uP central de falhas em seu funcionamento. A comunicação do RTC é realizada pelo protocolo I2C.

\item TC -- Timer/Counter

Em contrapartida ao RTC, o Timer não guarda informações sobre a data precisa atual, mas sim lida com uma
contagem de um dado tempo e é normalmente utilizado para gerar intervalos específicos marcados por tempo (um
exemplo análogo ao componente na vida real seria uma ampulheta).
Já o Counter, por sua vez, guarda a quantidade de vezes que um dado evento ocorrera dentro de um dado intervalo
de tempo, como rotações por minuto de um motor, requisições por minuto à um componente etc.
\end{itemize}

\subsection{Mapa de memória}
\label{sec:orgheadline2}
\begin{itemize}
\item Qual endereço de memória reservado para os periféricos?

Os endereços de memória de periféricos vão de 0x40000000 até 0x60000000.

\item Qual o tamanho (em endereço) dessa secção?

A secção possui 0x20000000 endereços possíveis.
\end{itemize}

\subsection{Encontre os endereços de memória referentes aos seguintes periféricos:}
\label{sec:orgheadline3}
\begin{itemize}
\item PIOA

0x400E0E00

\item PIOB

0x400E1000

\item ACC

0x40044000

\item UART1

0x400E0A00

\item UART2

0x400E1A00
\end{itemize}

\section{PMC - Gerenciador de energia}
\label{sec:orgheadline6}
\subsection{Qual o ID do PIOC?}
\label{sec:orgheadline5}
O ID do PIOC é 12

\section{Parallel Input Output (PIO)}
\label{sec:orgheadline11}
\subsection{Verifique quais periféricos podem ser configuráveis nos I/Os}
\label{sec:orgheadline7}
\begin{itemize}
\item PC1

O PC1 pode ser configurado como D1 ou PWMC0\(_{\text{PWML1}}\).

\item PB6

O PB6 funciona exclusivamente como SWDIO/TM5.
\end{itemize}

\subsection{Debouncing}
\label{sec:orgheadline8}
\begin{itemize}
\item O que é debouncing?

O debouncing é uma técnica de garantir programaticamente que um botão tenha sido apertado somente uma vez. A
necessidade do debouncing surge do fato que ao apertar um botão um sinal muito ruidoso é gerado, podendo, as
vezes, gerar um sinal análogo à um duplo clique, ou seja, filtrar o sinal de clique de um botão para garantir
que será computado apenas o verdadeiro clique no mesmo.

\item Descreva um algorítmo que implemente o debouncing.

Um algorítmo que implemente o debouncing poderia seguir assim:
Ao apertar o botão, segure sua thread por um pequeno intervalo, se após este pequeno intervalo o botão continuar
apertado, isso significa que sua leitura de fato fora de um único clique, caso contrário, sua leitura foi,
provavelmente, de algum ruído e deve ser ignorada.
\end{itemize}

\subsection{Race conditions}
\label{sec:orgheadline9}
\begin{itemize}
\item O que são race conditions?

Race conditions são falhas de concorrências em um sistema computacional, normalmente atrelada a não expectativa
da necessidade de um dado processo em outro.

\item Como que essa forma de conmfigurar registradores evita isso?

Ao permitir a mudança de dados de um registrador somente por um operador externo, evita-se que os dados do mesmo
sejam alterados como consequência em alguma outra operação do programa, tornando menos race conditions cada vez
menos prováveis.
\end{itemize}

\subsection{Explique com suas palavras o trecho anterior extraído do datasheet do uC, se possível referencie com um diagrama "I/O Line Control Logic".}
\label{sec:orgheadline10}
Para estabelecer a comunicação entre um periférico e um uC é necessário, primeiramente, definir qual dos dois será
o controlador da relação, ou seja, qual dos dois "enviará" dados para o outro. No SAM-E70, podemos definir um
periférico como controlador de uma comunicação ao "settar" seu respectivo pino de comunicação em 0. Por exemplo:
Ao definir o PIO\(_{\text{PSR}}\) como 0, seus respectivos periféricos serão os ativos na comunicação com o uC. Da mesma forma,
ao definir o valor do PIO\(_{\text{PSR}}\) como 1, definimos a comunicação como guiada pelo controlador. Neste caso, o
periférico escreve seu valor de saída mais recente em um registrador de output exclusivo para o mesmo (PIO\(_{\text{CODR}}\)).
\end{document}