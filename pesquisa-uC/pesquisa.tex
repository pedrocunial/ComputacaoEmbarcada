% Created 2017-02-07 Tue 13:53
\documentclass[11pt]{article}
\usepackage[utf8]{inputenc}
\usepackage[T1]{fontenc}
\usepackage{fixltx2e}
\usepackage{graphicx}
\usepackage{grffile}
\usepackage{longtable}
\usepackage{wrapfig}
\usepackage{rotating}
\usepackage[normalem]{ulem}
\usepackage{amsmath}
\usepackage{textcomp}
\usepackage{amssymb}
\usepackage{capt-of}
\usepackage{hyperref}
\author{Pedro Cunial}
\date{\today}
\title{Pesquisa de Microcontroladores - Computação Embarcada}
\hypersetup{
 pdfauthor={Pedro Cunial},
 pdftitle={Pesquisa de Microcontroladores - Computação Embarcada},
 pdfkeywords={},
 pdfsubject={},
 pdfcreator={Emacs 25.1.1 (Org mode 8.3.6)},
 pdflang={Portuguese}}
\begin{document}

\maketitle

\section{Visão Geral}
\label{sec:orgheadline1}
\begin{enumerate}
  \item Quais são os principais fabricantes de microcontrolador?

    Os três maiores fabricantes de microcontroladores são (sem nenhuma ordem específica) Texas, Atmel e Microchip.

  \item Liste os processadores utilizados por pelo menos 3 tipos de Arduino e faça
    um comparativo entre eles.

    O Arduino Uno possui um controlador ATmega328 (Atmel), o Mega possui um
    controlador ATmega2560 (Atmel), enquanto o Due possui um ARM Cortex-M3.
    Enquanto ambos processadores ATmega são 8 bits, o Cortex-M3 é de 32 bits e,
    por mais que todos sejam RISC, somente o Cortex-M3 é ARM, enquando os da Atmel
    são AVR. A diferença entre os processadores do Uno e do Mega é que o do Mega é
    mais potente.

  \item O que é bigendian e little endian (Endianness)?

    Little endian e bigendian são maneiras de se organizar dados que ocupariam
    mais de um byte (8 bits). O little endian ordena os bytes são guardados de
    forma que o os bytes que formam as menores casas do número ficam nas primeiras
    casas, enquanto o bigendian é o contrário. Os processadores da Intel (x86) e
    AMD (processadores mais comuns em computadores) usam a arquitetura little
    endian. Não existem exemplos atuais de processadores que utilizem
    exclusivamente o bigendian, no entanto, algumas arquiteturas, como a ARM,
    utilizam a chamada bi-endian, onde pode-se alterar a maneira com que se guarda
    os bytes, permitindo maior eficiência.

\end{enumerate}

\section{ARM}
\label{sec:orgheadline2}
\begin{enumerate}
  \item Descreva o funcionamento do barramento AMBA (APB, AHB) e como o mesmo é
    utilizado.

    O barramento AMBA (Advanced Microcontroller Bus Architecture) é utilizada
    como o padrão de comunicação interno de chips ARM. O barramento AMBA tem
    como principal vantagem a possibilidade de reuso de IP e melhor comunicação
    com o mesmo, tornando-se o largamente adotado pela industria.

    O design e principio de funcionamento do AMBA baseia-se ser o mais
    compatível possível com periféricos, além do reuso dos núcleos de IP
    (Intellecutal Property),
    reduzindo possíveis problemas envolvendo licensas em semicondutores.

  \item O que é ARM Thumb Instruction Set?

    O Thumb Instruction Set é uma maneira alternativa que os chips ARM aceitam
    de receber suas instruções, que tradicionalmente seriam de 32 bits, passam a
    ser de 16 bits. É importante ressaltar que o instruction set de uma
    aplicação pode ser definido pelo seu desenvolvedor, mas que uma vez
    definido, uma aplicação não poderá ter mais de um formato.

  \item O que é a Float Point Unit (FPU) e qual a sua utiização?

    A FPU é um hardware análogo à Unidade Lógica Aritmética, mas que trabalham
    com o cálculo exclusivo de números de ponto flutuante, inicialmente eram
    consideradas ``artigos de luxo'', mas ao final da década de 1990 já
    passariam à fazer parte das principais CPUs.

\end{enumerate}

\section{Tópicos extras}
\label{sec:orgheadline3}
\begin{enumerate}
  \item Qual a forma de medir desempenho de um uC?

    Apesar de existirem diversas maneiras de se medir o desempenho de um
    microcontrolador, a maneira mais utilizada é o ``tamanho da palavra'', ou
    seja o comprimento máximo de um número binário que este processador consegue
    manipular, por exemplo, processadores ARM tem um maior ``tamanho de
    palavra'' do que os Atmel utilizados nos Arduino Uno e Mega (enquanto os
    Atmel são de 8 bits, os ARM são de 32 bits).

  \item Quais são os modos de endereçamento de memória em um uC?

    De maneira genérica, a CPU se comunica com a memória por dois barramentos
    distintos: O barramento de endereços e o barramento de dados. Como o próprio
    nome já indica, é pelo barramento de endereços que a CPU indica onde deve-se
    ler ou escrever na memória do microcontrolador. A memória costuma ser
    alocada em blocos consecutivos, os quais podem ser sub-dividios de acordo
    com o tamanho de palavra do mesmo (vide pergunta anterior).

  \item Classifique os tipos de memórias de um uC.

    Existem três principais tipos de memórias em uCs, as memórias FLASH, EEPROM
    e SRAM. A memória FLASH é análoga ao HD de um computador, onde são
    armazenados dados de saída de programas. É relevante ressaltar que dados
    inseridos na FLASH não podem ser alterados pela execução de programas.

    A memória EEPROM é uma memória chamada não-volátil, de forma que, mesmo com
    o uC desligado os dados permanecem intactos. A EEPROM tem uma vida útil, por
    tanto não é recomendado que sejam armazenados dados que mudam com
    frequência, mas sim informações como valores de configuração e setup de
    diversos softwares ou até mesmo do SO.

    Por fim, a SRAM é a memória mais acessada e escrita pelo uC. Quando
    declaramos uma variável da maneira mais tradicional é lá que será
    armazenada, lida e tratada de maneira geral. Ao iniciar um programa, os
    dados do mesmo que estão na memória FLASH são copiados para a SRAM, onde o
    programa é de fato executado/manipulado.
\end{document}